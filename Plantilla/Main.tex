%% Version 7. Created January 2025.  
%%
%% AASTeX v7 calls the following external packages:
%% times, hyperref, ifthen, hyphens, longtable, xcolor, 
%% bookmarks, array, rotating, ulem, and lineno 
%%
%% RevTeX is no longer used in AASTeX v7.
%%
%% [MOD] s in .cls have been made. Look for them with the [MOD] tag.

%% [INPUT] - Change language as needed. Available languages: es, en

\documentclass[preprint2, es]{aastex7}

\makeatletter
\@ifclasswith{aastex7}{es}{
  % Spanish Definitions
  \def\partname{Parte}%
  \def\tocname{Contenidos}%
  \def\lofname{Lista de Figuras}%
  \def\lotname{Lista de Tablas}%
  \def\refname{Referencias}%
  \def\indexname{Índice}%
  \def\figurename{Figura}%
  \def\figuresname{Figuras}%
  \def\tablename{Tabla}%
  \def\tablesname{Tablas}%
  \def\abstractname{Resumen}%
  \def\appendixesname{Apéndices}%
  \def\appendixname{Apéndice}%
  \def\facilitiesname{Lugares}%
  \def\today{\ifcase\month\or
  Enero\or Febrero\or Marzo\or Abril\or Mayo\or Junio\or
  Julio\or Agosto\or Septiembre\or Octubre\or Noviembre\or Diciembre\fi
  \space\number\day, \number\year}
  \def\partname{Parte}
  \def\tocname{Contenidos}
  \def\lofname{Lista de Figuras}
  \def\lotname{Lista de Tablas}
  \def\refname{Referencias}
  \def\indexname{Índice}
  \def\figurename{Figura}
  \def\figuresname{Figuras}%
  \def\tablename{Tabla}
  \def\tablesname{Tablas}%
  \def\appendixesname{Apéndices}%
  \def\appendixname{Apéndice}%
  \def\acknowledgmentsname{Agreadecimientos}
  \def\journalname{??}
  \def\copyrightname{??}
  \def\andname{y}
  \def\ppname{pp}
  \def\numbername{número}
  \def\volumename{volumen}
  \def\Dated@name{Fecha }%
  \def\Received@name{Recibido }%
  \def\Revised@name{Revisado }%
  \def\Accepted@name{Aceptado }%
  \def\Published@name{Publicado }%
  \def\LabDateName{Fecha del Laboratorio}%
  \def\ReportDateName{Fecha del Informe}%
  \def\CommonOfDate{de}
  \def\SubjectTitleName{Asignatura}
  \def\assisttextsingular{Ayudante}
  \def\assisttextplural{Ayudantes}
  \def\professortextsingular{Profesor}
  \def\professortextplural{Profesores}
  \def\studenttextsingular{Integrantes}
  \def\studenttextplural{Integrantes}
}{
  % English Definitions (Default)
  \def\partname{Part}%
  \def\tocname{Content}%
  \def\lofname{Figures List}%
  \def\lotname{Tables List}%
  \def\refname{References}%
  \def\indexname{Index}%
  \def\figurename{Figure}%
  \def\figuresname{Figures}%
  \def\tablename{Table}%
  \def\tablesname{Tables}%
  \def\abstractname{ABSTRACT}%
  \def\appendixesname{Appendices}%
  \def\appendixname{Appendix}%
  \def\facilitiesname{Facilities}%
  \def\today{\ifcase\month\or
  January\or February\or March\or April\or May\or June\or
  July\or August\or September\or October\or November\or December\fi
  \space\number\day, \number\year}
  \def\partname{Part}
  \def\tocname{Contents}
  \def\lofname{List of Figures}
  \def\lotname{List of Tables}
  \def\refname{References}
  \def\indexname{Index}
  \def\figurename{Figure}
  \def\figuresname{Figures}%
  \def\tablename{Table}
  \def\tablesname{Tables}%
  \def\appendixesname{Appendixes}%
  \def\appendixname{Appendix}%
  \def\acknowledgmentsname{ACKNOWLEDGMENTS}
  \def\journalname{??}
  \def\copyrightname{??}
  \def\andname{and}
  \def\ppname{pp}
  \def\numbername{number}
  \def\volumename{volume}
  \def\Dated@name{Dated: }%
  \def\Received@name{Received }%
  \def\Revised@name{Revised }%
  \def\Accepted@name{Accepted }%
  \def\Published@name{Published }%
  \def\LabDateName{Lab Date}%
  \def\ReportDateName{Report Date}%
  \def\CommonOfDate{of}%
  \def\SubjectTitleName{Subject}%
  \def\assisttextsingular{Assistant}%
  \def\assisttextplural{Assistants}%
  \def\professortextsingular{Professor}%
  \def\professortextplural{Professors}%
  \def\studenttextsingular{Member}%
  \def\studenttextplural{Members}%
}
\makeatother


% ==============================================================================================================
% ================================================ INPUT =======================================================
% ==============================================================================================================


% ================================================ COVER =======================================================

% Logos located at Resources/logos/...
\def\coversubjectlogo{AST-h-black.png} 
\def\coversubjectlogoscale{0.6} 
\def\largetitlename{Laboratorio 1: Análisis de Espectros Estelares}

% Select wether you want to include ORCID iDs and links
\newif\iforcidbool
% \orcidboolfalse   % or 
\orcidbooltrue
\def\studenttextsingular{Integrantes}
\def\studenttextplural{Integrantes}

% --- Authors ---
\def\coverauthors{
    {0009-0005-4685-695X}{Juan José Sánchez Medina},
    {}{Henrietta Leavitt},
    {}{Stephen Hawking}
}

% --- Professors ---
\def\professortextsingular{Profesor}
\def\professortextplural{Profesores}
\def\coverprofessor{
    {0000-0000-0000-0000}{Isaac Newton}
}

% --- Assistants ---
\def\assisttextsingular{Ayudante}
\def\assisttextplural{Ayudantes}
\def\coverassist{
    {}{Marie Curie}
}

% --- Lab Date ---
\def\docinitdate{6 de noviembre}
% --- Final Date ---
\def\docfinalmonth{noviembre}

% --- Subject Details ---
\def\subjectcode{AST067}
\def\subjecttext{Laboratorio de Astronomía}


% ================================================ HEADER ======================================================
\shorttitle{Informe Laboratorio 01}
\shortauthors{Grupo 2070}

%% Include dates for submitted, revised, and accepted.
%%\received{February 1, 2025}
%%\revised{March 1, 2025}
%%\accepted{\today}
%%
%% Indicate AAS Journal the manuscript was submitted to.
%%\submitjournal{PSJ}
%% Note that this command adds "Submitted to " the argument.
%%
%% You can add a light gray and diagonal water-mark to the first page 
%% with this command:
%% \watermark{text}
%% where "text", e.g. DRAFT, is the text to appear.  If the text is 
%% long you can control the water-mark size with:
%% \setwatermarkfontsize{dimension}
%% where dimension is any recognized LaTeX dimension, e.g. pt, in, etc.
%%%%%%%%%%%%%%%%%%%%%%%%%%%%%%%%%%%%%%%%%%%%%%%%%%%%%%%%%%%%%%%%%%%%%%%%%%%%%%%%
%%
%% Use this command to indicate a subdirectory where figures are located.
%%\graphicspath{{./}{figures/}}

% =============================================== PREAMBLE =======================================================

% --- Custom Imports ---
\usepackage{amsmath}
\usepackage{cancel}
\usepackage{lipsum}
\usepackage{multirow}
\usepackage{soul}
\usepackage{todonotes}

% --- Custom Commands ---
\newcommand{\vdag}{(v)^\dagger}
\newcommand\aastex{AAS\TeX}
\newcommand\latex{La\TeX}
\newcommand{\diff}{\mathop{}\!\mathrm{d}}
\newcommand{\cint}{%
  \mathop{%
    \vcenter{\hbox{\(\displaystyle\int\mkern-18mu C\)}}%
  }\nolimits%
}


\makeatletter
\let\frontmatter@title@above=\relax
\makeatother

%%%%%%%%%%%%%%%%%%%%%%%%%%%%%%%%%%%%%%%%%%%%%%%%%%%%%%%%%%%%%%%%%%%%%%%%%%%%%%%%
%%
%% The following section outlines numerous optional output that
%% can be displayed in the front matter or as running meta-data.
%%
%% Running header information. A short title on odd pages and 
%% short author list on even pages. Note that this
%% information may be modified in production.



%%
%% This initial command takes arguments that can be used to easily modify 
%% the output of the compiled manuscript. Any combination of arguments can be 
%% invoked like this:
%%
%% \documentclass[argument1,argument2,argument3,...]{aastex7}
%%
%% Six of the arguments are typestting options. They are:
%%
%%  twocolumn   : two text columns, 10 point font, single spaced article.
%%                This is the most compact and represent the final published
%%                derived PDF copy of the accepted manuscript from the publisher
%%  default     : one text column, 10 point font, single spaced (default).
%%  manuscript  : one text column, 12 point font, double spaced article.
%%  preprint    : one text column, 12 point font, single spaced article.  
%%  preprint2   : two text columns, 12 point font, single spaced article.
%%  modern      : a stylish, single text column, 12 point font, article with
%% 		  wider left and right margins. This uses the Daniel
%% 		  Foreman-Mackey and David Hogg design.
%%
%% Note that you can submit to the AAS Journals in any of these 6 styles.
%%
%% There are other optional arguments one can invoke to allow other stylistic
%% actions. The available options are:
%%
%%   astrosymb    : Loads Astrosymb font and define \astrocommands. 
%%   tighten      : Makes baselineskip slightly smaller, only works with 
%%                  the twocolumn substyle.
%%   times        : uses times font instead of the default.
%%   linenumbers  : turn on linenumbering. Note this is mandatory for AAS
%%                  Journal submissions and revisions.
%%   trackchanges : Shows added text in bold.
%%   longauthor   : Do not use the more compressed footnote style (default) for 
%%                  the author/collaboration/affiliations. Instead print all
%%                  affiliation information after each name. Creates a much 
%%                  longer author list but may be desirable for short 
%%                  author papers.
%% twocolappendix : make 2 column appendix.
%%   anonymous    : Do not show the authors, affiliations, acknowledgments,
%%                  and author contributions for dual anonymous review.
%%  resetfootnote : Reset footnotes to 1 in the body of the manuscript.
%%                  Useful when there are a lot of authors and affiliations
%%		    in the front matter.
%%   longbib      : Print article titles in the references. This option
%% 		    is mandatory for PSJ manuscripts.
%%
%% Since v6, AASTeX has included \hyperref support. While we have built in 
%% specific %% defaults into the classfile you can manually override them 
%% with the \hypersetup command. For example,
%%
%% \hypersetup{linkcolor=red,citecolor=green,filecolor=cyan,urlcolor=magenta}
%%
%% will change the color of the internal links to red, the links to the
%% bibliography to green, the file links to cyan, and the external links to
%% magenta. Additional information on \hyperref options can be found here:
%% https://www.tug.org/applications/hyperref/manual.html#x1-40003
%%
%% The "bookmarks" has been changed to "true" in hyperref
%% to improve the accessibility of the compiled pdf file.
%%
%% If you want to create your own macros, you can do so
%% using \newcommand. Your macros should appear before
%% the \begin{document} command.
%%

%% This is the end of the preamble.

% >>>>>>>>>>>>>>>>>>>>>>>>>>>>>>>>>>>>>>> BEGIN DOCUMENT <<<<<<<<<<<<<<<<<<<<<<<<<<<<<<<<<<<<<<<<<<<<<<<<<<<<<<<<<
\begin{document}

\makeatletter
\newcount\authorcount
\def\countauthors{%
    \authorcount=0%
    \@for\@temp:=\coverauthors\do{\advance\authorcount by 1}%
}

% Main
\newcommand{\printauthors}{%
    \countauthors%
    \begin{center}
    \large\textbf{%
        \ifnum\authorcount>1%
            \studenttextplural%
        \else%
            \studenttextsingular%
        \fi%
    }
    \end{center} \vspace{-0.35cm}
    \expandafter\@processauthors\coverauthors,\@nil%
}

% Process author list
\def\@processauthors#1,#2\@nil{%
    \ifx&#1&%
    \else%
        \expandafter\@processauthor#1\\%
        \ifx&#2&%
        \else%
            \@processauthors#2\@nil%
        \fi%
    \fi%
}

\def\@processauthor#1#2\\{%
    \href{https://orcid.org/#1}{\textsc{#2} \includegraphics[height=0.8em]{Resources/misc/orcid-ID.png}}\\
}
\makeatother

\makeatletter
\newcount\authorcount
\def\countauthors{%
    \authorcount=0%
    \@for\@temp:=\coverprofessor\do{\advance\authorcount by 1}%
}

% Main
\newcommand{\printprof}{%
    \countauthors%
    \begin{center}
    \large\textbf{%
        \ifnum\authorcount>1%
            \professortextplural%
        \else%
            \professortextsingular%
        \fi%
    }
    \end{center} \vspace{-0.35cm}
    \expandafter\@processauthors\coverprofessor,\@nil%
}

% Process author list
\def\@processauthors#1,#2\@nil{%
    \ifx&#1&%
    \else%
        \expandafter\@processauthor#1\\%
        \ifx&#2&%
        \else%
            \@processauthors#2\@nil%
        \fi%
    \fi%
}

\def\@processauthor#1#2\\{%
    \href{https://orcid.org/#1}{\textsc{#2} \includegraphics[height=0.8em]{Resources/misc/orcid-ID.png}}\\
}
\makeatother



\begin{titlepage}
    \newcommand{\HRule}{\rule{\linewidth}{0.5mm}}
    \center
    % \textsc{\LARGE Pontificia Universidad Católica de Chile}\\[1cm] 
    
    \vspace{1.5cm}
    % If you want to override the logo, uncomment the line below and use and comment \coverlogo
    % \includegraphics[height=6cm]{Resources/logos/AST-center-black.png}
    \includegraphics[scale=\coversubjectlogoscale]{Resources/logos/\coversubjectlogo}
    
    \vspace{1cm}
    \HRule \\[0.4cm]
    {\huge \bfseries \largetitlename}\\
    \HRule \\
    
    \vfill
    
    \large \textbf{Asignatura} \\
    \textsc{AST101} - Objetos Messier
    
    \vspace{1cm}
    
    % \large \textbf{Estudiante} \\
    \printauthors
    
    \vspace{1cm}
    
    \printprof
    
    \vfill
    
    \begin{minipage}[t]{0.45\textwidth}
        \centering
        \large \textbf{Fecha de la visita} \\ 24 de Abril, \the\year
    \end{minipage}
        \hfill
    \begin{minipage}[t]{0.45\textwidth}
        \centering
        \large \textbf{Fecha de entrega} \\ \the\day \ de Abril, \the\year
    \end{minipage}
    
\end{titlepage}

% ================================================ TITLE =======================================================
% \title{\Large{\largetitlename}}
\title{\Large{}}


% ================================================ AUTHORS =====================================================
%% A significant change from AASTeX v6+ is in the author blocks. Now an email
%% address is required for each author. This means that each author requires
%% at least one of the following:
%%
% \author
% \affiliation
% \email

%% If these three commands are not available for each author, the latex
%% compiler will issue an error and if you force the latex compiler to continue,
%% it will generate an incomplete pdf.
%%
%% Multiple \affiliation commands are allowed and authors can also include
%% an optional \altaffiliation to indicate a status, i.e. Hubble Fellow. 
%% while affiliations are indexed as footnotes, altaffiliations are noted with
%% with a non-numeric footnote that is set away from the numeric \affiliation 
%% footnotes. NOTE that if an \altaffiliation command is used it must 
%% come BEFORE the \affiliation call, right after the \author command, in 
%% order to place the footnotes in the proper location. Because non-numeric
%% symbols are used, \altaffiliation should be used sparingly.
%%
%% In v7 the \author command takes an optional argument which provides 
%% additional metadata about the author. Authors can provide the 16 digit 
%% ORCID, the surname (family or last) name, the given (first or fore-) name, 
%% and a name suffix, e.g. "Jr.". The syntax is:
%%
%% \author[orcid=0000-0002-9072-1121,gname=Gregory,sname=Schwarz]{Greg Schwarz}
%%
%% This name metadata in not shown, it is only for parsing by the peer review
%% system so authors can be more easily identified. This name information will
%% also be sent to the publisher so they can include it in the CROSSREF 
%% metadata. Including an orcid will hyperlink the author name to the 
%% author's ORCID page. Note that  during compilation, LaTeX will do some 
%% limited checking of the format of the ID to make sure it is valid. If 
%% the "orcid-ID.png" image file is  present or in the LaTeX pathway, the 
%% ORCID icon will appear next to the authors name.
%%
%% Even though emails are now required for each author, the \email does not
%% produce output in the compiled manuscript unless the optional "show" command
%% is used. For example,
%%
% \email[show]{greg.schwarz@aas.org}
%%
%% All "shown" emails are show in the bottom left of the first page. Due to
%% space constraints, only a few emails should be shown. 
%%
%% To identify a corresponding author, use the \correspondingauthor command.
%% The command appends "Corresponding Author: " to the argument it appears at
%% the bottom left of the first page like the output from \email. 

% \author[orcid=0009-0005-4685-695X,sname='Pogson']{Norman Pogson}
\affiliation{Pontificia Universidad Católica de Chile}
\email[show]{isaac.newton@estudiante.uc.cl}  


% \collaboration{all}{The Terra Mater collaboration}

%% Use the \collaboration command to identify collaborations. This command
%% takes an optional argument that is either a number or the word "all"
%% which tells the compiler how many of the authors above the command to
%% show. For example "\collaboration[all]{(DELVE Collaboration)}" wil include
%% all the authors above this command.
%%
%% Mark off the abstract in the ``abstract'' environment. 

% ================================================ ABSTRACT ======================================================

\begin{abstract}
    En este informe se va a detallar la visita realizada al Observatorio UC (OUC) como
    parte del curso \textsc{AST111}. Adicionalmente se contestarán las preguntas de la 
    \href{https://cloudia.astro.puc.cl/nextcloud/index.php/s/aHYxZcoCFjd2GLg?dir=/&openfile=true}{Guía A}.

\end{abstract}

% ================================================ KEYWORDS ======================================================
\input{Keywords.tex}

%% From the front matter, we move on to the body of the paper.
%% Sections are demarcated by \section and \subsection, respectively.
%% Observe the use of the LaTeX \label
%% command after the \subsection to give a symbolic KEY to the
%% subsection for cross-referencing in a \ref command.
%% You can use LaTeX's \ref and \label commands to keep track of
%% cross-references to sections, equations, tables, and figures.
%% That way, if you change the order of any elements, LaTeX will
%% automatically renumber them.

% =============================================== MAIN CONTENT ====================================================

\section{Objetivos e Hipótesis} \label{sec:objetivos}

\lipsum[1]
\section{Introducción y Marco Teórico} \label{sec:intro}

\subsection{Energía} \label{subsec:energy}

\lipsum[1] \citep{resnick_physics}.

\subsubsection{Energía Cinética} \label{subsubsec:kinetic}

\lipsum[1] \ref{ec:kinetic_e}. Donde \(m\) es la masa del carro y \(v\) su velocidad \citep{resnick_physics}.

\begin{equation}
    \label{ec:kinetic_e}
    K = \frac{1}{2}mv^2
\end{equation}

\subsubsection{Energía Potencial Gravitatoria} \label{subsubsec:potential}

\lipsum[1][1] \ref{ec:mgh}. \lipsum[1][1] (\(m\)) y la altura (\(h\)) del cuerpo \citep{Kleppner}.  

\begin{equation}
    \label{ec:mgh}
    U = mgh
\end{equation}

\lipsum[1][2] \citep{resnick_physics}.

\subsubsection{Energía Mecánica} \label{subsubsec:mechanic_e}

\lipsum[1][3] \ref{ec:total_e} \citep{resnick_physics}.

\begin{equation}
    \label{ec:total_e}
    E = K + U
\end{equation}

\lipsum[1][1]

\subsubsection{Conservación de la Energía} \label{subsubsec:energy_conservation}

\lipsum[1][1] \citep{Kleppner}.

\begin{equation}
    \label{ec:conservation_e}
    K_i + U_i = K_f + U_f = E
\end{equation}

\lipsum[1][1]

\begin{figure}[ht!]
    \centering
    \includegraphics[width=8.5cm]{Resources/img/energy_linear_unitless.png}
    \caption{Ejemplo de conservación de la energía. Los datos se encuentran normalizados.}
    \label{fig:energy_linear}
\end{figure}

\subsection{Fuerzas} \label{subsec:forces}

\subsubsection{Teorema Trabajo-Energía} \label{subsubsec:work_energy_teo}

\lipsum[1] \citep{Kleppner}. Este teorema permite analizar sistemas dinámicos considerando únicamente las fuerzas que actúan sobre estos.

\begin{equation}
    \label{ec:work_energy}
    W_{b\to a} = \cint^{\hspace{0.1cm}a}_{\hspace{-0.2cm}b} \mathbf{F} \cdot \diff \mathbf{r}
\end{equation}

\begin{equation}
    \label{ec:work_energy_2}
    W_{b \to a} = \Delta E
\end{equation}

\lipsum[1][1]


\subsubsection{Movimiento Circular} \label{subsubsec:circular_mov}

\lipsum[2]
\begin{equation}
    \label{ec:c_accel}
    \mathbf{a}_c = \frac{v^2}{r} \, \hat{\mathbf{r}}
\end{equation}

\lipsum[1]

\begin{equation}
    \label{ec:f_c}
    \mathbf{F}_c = m \mathbf{a}_c = m \frac{v^2}{r} \, \hat{\mathbf{r}}
\end{equation}

\lipsum[1]

\begin{equation}
\label{ec:f_cf}
    \mathbf{F}_{cf} = - m \frac{v^2}{r} \, \hat{\mathbf{r}}
\end{equation}

 
\section{Montaje Experimental} \label{sec:montaje_ex}

\lipsum[2]

\subsection{Montaje}

\begin{figure*}[ht!]
    \centering
    \includegraphics[width=12cm]{Resources/img/setup.png}
    \caption{Montaje Montaña Rusa.}{\lipsum[1][1]}
    \label{fig:setup}
\end{figure*}

\lipsum[2]


\begin{figure}[ht!]
    \centering
    \includegraphics[width=5cm]{Resources/img/rcm.png}
    \caption{Carro usado.}{1: Carro plástico y 2: Masa adicional.}
    \label{fig:rcm}
\end{figure}

\lipsum[1]

\begin{align}
    m_{tot} &= m_c + m_e \notag \\
    &= 12309.12 \ \text{g} + 123.2 \ \text{g} \notag \\
    m_{tot} &= 1000.04 \ \text{g} \label{ec:mtot}
\end{align}

\lipsum[1]


\subsection{Procedimiento}

\lipsum[2]
\section{Resultados y Análisis} \label{sec:resultados}

\lipsum[1]

\begin{equation}
\label{ec:analysis_1}
\frac{1}{2} m v_\text{top}^2 + m g h_\text{top} = m g h_\text{rel}
\end{equation}

\lipsum[1][1] \ref{ec:analysis_1} \lipsum[1][1] \ref{ec:v_top} \lipsum[1][1] \textit{loop}.

\begin{equation}
\label{ec:v_top}
v_\text{top} = \sqrt{2 g (h_\text{rel} - 2 \bar{r})}
\end{equation}

\lipsum[1][2] \ref{ec:eq_forces} \lipsum[1][2] \textit{loop}.

\begin{equation}
\label{ec:eq_forces}
\frac{m v_\text{top}^2}{\bar{r}} \ge m g \quad \Rightarrow \quad v_\text{top} \ge \sqrt{\bar{r}g},
\end{equation}

\lipsum[1][3] \ref{ec:v_top} y \ref{ec:eq_forces}, tomando a \(h_\text{top} = 2 \bar{r}\), se obtiene la altura mínima teórica de lanzamiento necesaria para completar el \textit{loop} expresada por la ecuación \ref{ec:min_h}.

\begin{equation}
\label{ec:min_h}
h_\text{min} = \frac{5}{2} \bar{r}
\end{equation}

\lipsum[1]

\begin{figure}[ht!]
    \centering
    \includegraphics[width=8cm]{Resources/img/dcl.png}
    \caption{\lipsum[1][1]}
    \label{fig:dcl}
\end{figure}

\lipsum[1][3] \(h_{min} = 12399.01 \ \text{cm}\). \lipsum[1][3] \(h^{min}_{exp} = 12093.50 \ \text{cm}\).

\begin{deluxetable}{cccccc}[ht!]
\tablewidth{0pt}
\tablecaption{\lipsum[1][1] \label{tab:results}}
\tablehead{
\colhead{\parbox[c]{1.2cm}{\centering \(h_\text{rel}\) \\ (cm)}} &
\colhead{\parbox[c]{1cm}{\centering $\bar{v}_\text{exp}$ \\ (m/s)}} &
\colhead{\parbox[c]{1.5cm}{\centering $\sigma_{v_\text{exp}}$ \\ (m/s)}} &
\colhead{\parbox[c]{1cm}{\centering $v_\text{teo}$ \\ (m/s)}} &
\colhead{\parbox[c]{1.5cm}{\centering $v_\text{error}$ \\ (\%)}} 
}
\startdata
3.21 & 111 & 0.221 & 120.2 & 12.32 \\
3.21 & 111 & 0.221 & 120.2 & 12.32 \\
3.21 & 111 & 0.221 & 120.2 & 12.32 \\
3.21 & 111 & 0.221 & 120.2 & 12.32 \\
3.21 & 111 & 0.221 & 120.2 & 12.32 \\
\enddata
\tablecomments{\lipsum[1][2]}
\end{deluxetable}


\lipsum[1]

\begin{align}
    {v_\text{top}}^2 &= 2g(h_\text{rel} - 2\bar{r}) \notag \\
    {v_\text{top}}^2 &= 2gh_\text{rel} - 4g\bar{r} \notag \\
    \left({v_\text{top}(h_\text{rel})}\right)^2 &= \underbrace{2g}_{m}(h_\text{rel}) - \underbrace{4g\bar{r}}_{b} \label{ec:v_top_lineal}
\end{align}

\lipsum[1]

\begin{figure}[ht!]
    \centering
    \includegraphics[width=8cm]{Resources/img/exp_vs_teo_summary_sq_fit.png}
    \caption{\lipsum[1][1]}
    \label{fig:vel_normal}
\end{figure}

\lipsum[1]


\subsection{Análisis de Errores} \label{subsec:errors}

\lipsum[1]


\begin{figure}[ht!]
    \centering
    \includegraphics[width=8.5cm]{Resources/img/mountain_coaster_energy_combined_2.png}
    \caption{\lipsum[1][1]}
    \label{fig:comparision}
\end{figure}

\lipsum[1]
\section{Discusión} \label{sec:Discucines}


\sethlcolor{orange!30}
\hl{TODO: incluir discusión de resultados y análisis de errores.}

\lipsum[1]
\section{Conclusiones} \label{sec:conclusiones}

\lipsum[1]
\section{Preguntas} \label{sec:preguntas}

\subsection{Pregunta 1} \label{subsec:pregunta1}

\noindent \textbf{Pregunta} 

``Id nostrud cillum culpa et duis duis."

\noindent \textbf{Respuesta} 


Aliquip aliqua pariatur aliqua duis sint nostrud ex ipsum pariatur exercitation adipisicing in. Do culpa amet in minim et elit nostrud sit Lorem do fugiat veniam. Pariatur sint elit occaecat dolor minim adipisicing eiusmod deserunt ut id aliquip. Duis proident occaecat cupidatat consectetur laborum aute enim culpa. Duis eiusmod esse laborum ut ad reprehenderit nisi. Fugiat incididunt in labore eiusmod eu adipisicing tempor ut magna occaecat.
Incididunt sunt cupidatat in fugiat in velit non eu in reprehenderit ullamco. Ad elit irure eu enim adipisicing consectetur minim do duis excepteur tempor. Anim consectetur nostrud dolor deserunt ullamco do culpa velit. Nisi elit veniam pariatur tempor proident nulla sunt ad consequat commodo excepteur ex. Veniam voluptate quis irure veniam nisi duis eu ex deserunt anim officia sit proident. \href{https://www.slooh.com/}{Slooh}

Ullamco duis ipsum laborum dolore cupidatat ipsum. Exercitation eu cillum laborum deserunt et Lorem consequat fugiat Lorem duis officia proident. Officia laboris dolore non consectetur ut aliquip minim sint sint. Lorem aliquip pariatur minim pariatur consequat cillum minim minim. Velit ea aliquip in cillum excepteur sunt proident fugiat. Ipsum aliquip ullamco ad culpa et ea eiusmod cillum irure cillum.

Elit nisi qui minim officia veniam dolore exercitation do in dolor tempor. Proident nulla do cillum ipsum labore est adipisicing nulla. Nisi aliqua sit proident pariatur irure sunt dolore ipsum et cupidatat irure. Nostrud amet voluptate qui labore adipisicing qui culpa Lorem sint consectetur Lorem Lorem labore amet. Laborum qui ullamco dolor excepteur mollit. Nulla ut ad sint adipisicing dolore tempor reprehenderit id laborum laborum id. Proident exercitation minim nisi ut anim esse in ea sunt enim.

\begin{enumerate}
    \item Aute proident ad ex aliqua aliquip.
    \begin{itemize}
        \item \textbf{Sirius:} Ad Lorem fugiat ut fugiat ullamco quis fugiat adipisicing ut velit excepteur cupidatat. $\theta = 402 \ \text{mm}$
        \item \textbf{Astropy:} Adipisicing laboris excepteur laboris nostrud sint. \texttt{hello world}
    \end{itemize}

    \item Aliqua ad ullamco eiusmod id mollit fugiat proident sint incididunt sint exercitation minim irure.
    \begin{itemize}
        \item Irure aliquip et adipisicing deserunt consectetur.
        \item Exercitation consectetur ut elit veniam aute.
    \end{itemize}

    \item Hadrones \\

\end{enumerate}

Mollit et nisi nulla dolore ad nostrud officia. \object{M42} Ex velit laborum sint ullamco pariatur esse nulla enim sint magna minim aliqua ut excepteur. Excepteur velit duis voluptate id commodo consequat incididunt culpa sint ea in quis dolore magna. Officia cupidatat qui reprehenderit velit nulla quis sit. Elit proident eu duis veniam occaecat elit consequat et.

Aute fugiat cupidatat laboris esse elit aliquip veniam. Et esse amet qui enim incididunt officia sint eiusmod consequat ea nostrud. Do velit laborum exercitation officia.

\subsection{Pregunta 2} \label{subsec:pregunta2}
\noindent \textbf{Pregunta} 

``Incididunt ut excepteur amet in.''

\noindent \textbf{Respuesta}

Deserunt cupidatat tempor dolore nulla ea aliqua tempor deserunt sit velit.

% \begin{figure*}[ht!]
%     \centering
%     \includegraphics[width=12cm]{Resources/img/image.png}
%     \caption{Fotografía realizada en Stellarium}{Detalles: Tiempo de exposición $t = 45 \ \text{yrs}$}\\{JD1293719823.2}.
%     \label{fig:constelaciones}
% \end{figure*}

Do et irure do pariatur elit amet. \ref{fig:constelaciones} Et aute proident duis in elit eiusmod aliquip ipsum ullamco do voluptate.


\subsection{Pregunta 3} \label{subsec:pregunta3}

\noindent \textbf{Pregunta} 

Duis pariatur velit amet enim amet ea irure ullamco laborum nulla enim voluptate.

\begin{enumerate}
    \item[a)] Duis pariatur velit amet enim amet ea irure ullamco laborum nulla enim voluptate
    \item[b)] Duis pariatur velit amet enim amet ea irure ullamco laborum nulla enim voluptate
    \item[c)] Duis pariatur velit amet enim amet ea irure ullamco laborum nulla enim voluptate
\end{enumerate}

\noindent \textbf{Desarrollo} 

Duis pariatur velit amet enim amet ea irure ullamco laborum nulla enim voluptate.


\begin{figure*}[ht!]
    \gridline{\fig{Resources/tikz/output/all.pdf}{0.4\textwidth}{Solución a)}
              \fig{Resources/tikz/output/lat.pdf}{0.4\textwidth}{Solución b)}}
    \gridline{\fig{Resources/tikz/output/alt.pdf}{0.4\textwidth}{Solución c)}
              \fig{Resources/tikz/output/all.pdf}{0.4\textwidth}{Vista General}}
    \centering
    \caption{Eiusmod proident est labore occaecat cupidatat reprehenderit ut amet proident do consequat sunt. \ref{subsec:pregunta3}}
    {Esse in fugiat velit voluptate duis nulla.}
    \label{fig:pregunta3desarrollo}
\end{figure*}


\subsection{Pregunta 4} \label{subsec:pregunta4}
\noindent \textbf{Pregunta} 

``Veniam dolore aliqua ea mollit sunt aute esse."

\noindent \textbf{Respuesta}

Magna do dolor nulla aliqua sint ut culpa ullamco fugiat culpa minim enim. Non reprehenderit fugiat aliqua fugiat enim elit officia id sint. Duis amet ipsum dolor deserunt voluptate dolore incididunt laboris enim id elit aliqua id. Fugiat officia amet dolore proident proident.

Quis officia sit aliqua sint. Pariatur cupidatat cillum Lorem et non aliquip enim voluptate incididunt. Est consequat dolore minim laboris ut mollit laboris. Occaecat occaecat exercitation dolore laboris ad exercitation velit enim ut elit duis anim sint. Et dolore pariatur nostrud cillum sit ex minim.


\noindent \textbf{Pregunta}
``Quis sit labore elit ipsum excepteur.''

\noindent \textbf{Respuesta}
Eu ex ad deserunt laborum aute laborum culpa enim ipsum eu nisi. Ex enim id consequat in ex in velit pariatur tempor. Ipsum aute aliqua enim ut est dolore aliquip ipsum sint proident ut ad. Aliquip ex minim ex pariatur veniam tempor pariatur ea reprehenderit Lorem nisi ex enim.

Quis eiusmod nisi dolore proident labore sit ipsum minim ad veniam magna ea nisi. Officia et commodo ea eu in. Occaecat ullamco eiusmod sint occaecat. Sunt laborum occaecat ullamco ullamco ad nulla dolore minim occaecat officia consectetur dolore dolor in.

Et occaecat officia anim fugiat aliqua commodo et anim minim nulla tempor ipsum anim laborum. Esse proident reprehenderit ad proident consectetur officia labore nulla deserunt non ex. Occaecat incididunt do ad id sit aliqua. Tempor esse esse irure anim cillum sunt non ipsum elit mollit veniam. Elit quis ex Lorem cupidatat eu occaecat enim sint. Non ut aliqua elit enim labore ut nostrud culpa elit labore. Do laborum consequat occaecat est minim labore proident consectetur mollit aute ut.

Nostrud eu fugiat adipisicing incididunt. Eiusmod commodo pariatur esse sunt occaecat ullamco excepteur sint irure. Pariatur esse ad consequat ut aliqua in ut culpa consequat. Sint Lorem occaecat exercitation velit ad irure duis aute deserunt. Ad ipsum id cillum eu excepteur do laboris.

% Planck's Law Derivation (Equations Only with \[ \])

\[
E_n = n h \nu \quad \text{with} \quad n = 0, 1, 2, \dots
\]

\[
\langle E \rangle = \frac{\sum_{n=0}^{\infty} E_n e^{-E_n / k_B T}}{\sum_{n=0}^{\infty} e^{-E_n / k_B T}}
\]

\[
\langle E \rangle = \frac{\sum_{n=0}^{\infty} n h \nu e^{-n h \nu / k_B T}}{\sum_{n=0}^{\infty} e^{-n h \nu / k_B T}}
\]

\[
\langle E \rangle = \frac{h \nu}{e^{h \nu / k_B T} - 1}
\]

\[
u(\nu, T) = \frac{8 \pi \nu^2}{c^3} \langle E \rangle
\]

\[
u(\nu, T) = \frac{8 \pi h \nu^3}{c^3} \cdot \frac{1}{e^{h \nu / k_B T} - 1}
\]

\[
u(\lambda, T) = \frac{8 \pi h c}{\lambda^5} \cdot \frac{1}{e^{h c / \lambda k_B T} - 1}
\]




% ====================================== ACKNOWLEDGMENT & CONTRIBUTIONS ==========================================
%% Please use the acknowledgment and contribution environments. This will 
%% be anonomyized when the "anonymous" style option is used. 

\begin{acknowledgments}[Comentarios]

\end{acknowledgments}

% \begin{contribution}
    %%This section gives authors the space to recognize author contributions. The text inside this environment is NOT counted towards the total word quanta. At a minimum, manuscripts are expected to include this text:
    
    All authors contributed equally to the Terra Mater collaboration.
    
    %% But authors are expected to provide more specific details, e.g. 
    %%
    %%SC was responsible for writing and submitting the manuscript.
    %%WWM came up with the initial research concept and edited the manuscript.
    %%OTS obtained the funding and edited the manuscript.
    %%EBF provided the formal analysis and validation. He also edited the manuscript.
    %%GEH Supervised the undergraduates, wrote the software and administers the project github and Zenodo repositories.
    %%
    %% Authors can use the Contributor Role Taxonomy (CRediT) at
    %% https://credit.niso.org
    %% for ideas on how write a good statement tailored to their needs.
    
    \end{contribution}

% ============================================== FACILITIES ======================================================
%% To help institutions obtain information on the effectiveness of their 
%% telescopes the AAS Journals has created a group of keywords for telescope 
%% facilities.
%
%% Following the acknowledgments section, use the following syntax and the
%% \facility{} or \facilities{} macros to list the keywords of facilities used 
%% in the research for the paper.  Each keyword is check against the master 
%% list during copy editing.  Individual instruments can be provided in 
%% parentheses, after the keyword, but they are not verified.

% \facility{}
\facilities{Laboratorio Física UC}

% ================================================ SOFTWARE ======================================================
%% Similar to \facility{}, there is the optional \software command to allow 
%% authors a place to specify which programs were used during the creation of 
%% the manuscript. Authors should list each code and include either a
%% citation or url to the code inside ()s when available.

% Anadir sklearn, scipy.stats
\software{
    Python - Jupyter Notebook \citep{jupyter}, SciPy (Python) \citep{scipy}, Sk-learn (Python) \citep{scikit-learn}, 
    \LaTeX \citep{1989BAAS...21..780H}, \TeX \ Template \citep{Sanchez_Medina_AST-FIZ_TeX_Template_2025}.
}

\softwarecode{
    The code used to perform the analysis in this work is available at: \url{https://github.com/jj-sm}{GitHub}
}

% ================================================= APENDIX ======================================================
%% Appendix material should be preceded with a single \appendix command.
%% There should be a \section command for each appendix. Mark appendix
%% subsections with the same markup you use in the main body of the paper.
%%
%% Each Appendix (indicated with \section) will be lettered A, B, C, etc.
%% The equation counter will reset when it encounters the \appendix
%% command and will number appendix equations (A1), (A2), etc. The
%% Figure and Table counter will not reset.

% \appendix

% \section{Appendix information}

Appendices can be broken into separate sections just like in the main text.
The only difference is that each appendix section is indexed by a letter
(A, B, C, etc.) instead of a number.  Likewise numbered equations have
the section letter appended.  Here is an equation as an example.
\begin{equation}
I = \frac{1}{1 + d_{1}^{P (1 + d_{2} )}}
\end{equation}
Appendix tables and figures should not be numbered like equations. Instead
they should continue the sequence from the main article body.
% \section{Author publication charges} \label{sec:pubcharge}

In April 2011 the traditional way of calculating author charges based on 
the number of printed pages was changed.  The reason for the change
was due to a recognition of the growing number of article items that could not 
be represented in print. Now author charges are determined by a number of
digital ``quanta''.  A single quantum is defined as 350 words, one figure, one table,
and one digital asset.  For the latter this includes machine readable
tables, data behind a figure, figure sets, animations, and interactive figures.  The current cost
for the different quanta types is available at 
\url{https://journals.aas.org/article-charges-and-copyright/#author_publication_charges}. 
Authors may use the ApJL length calculator to get a {\tt rough} estimate of 
the number of word and float quanta in their manuscript. The calculator 
is located at \url{https://authortools.aas.org/ApJL/betacountwords.html}.
% \section{Rotating tables} \label{sec:rotate}

To place a single page table in a landscape mode start the table portion with {\tt\string\begin\{rotatetable\}} and end with {\tt\string\end\{rotatetable\}}.

Tables that exceed a print page take a slightly different environment since both rotation and long table printing are required. In these cases start with {\tt\string\begin\{longrotatetable\}} and end with {\tt\string\end\{longrotatetable\}}. The {\tt\string\movetabledown} command can be used to help center extremely wide, landscape tables. The command {\tt\string\movetabledown=1in} will move any rotated table down 1 inch. 

A handy "cheat sheet" that provides the necessary \latex\ to produce 17
different types of tables is available at \url{http://journals.aas.org/authors/aastex/aasguide.html#table_cheat_sheet}.

% \section{Using Chinese, Japanese, and Korean characters}

Authors have the option to include names in Chinese, Japanese, or Korean (CJK)
characters in addition to the English name. The names will be displayed
in parentheses after the English name. The way to do this in AASTeX is to
use the CJK package available at \url{https://ctan.org/pkg/cjk?lang=en}.
Further details on how to implement this and solutions for common problems,
please go to \url{https://journals.aas.org/nonroman/}.


% ============================================== BIBLIOGRAPHY ====================================================
%% For this sample we use BibTeX plus aasjournalv7.bst to generate the
%% the bibliography. The sample7.bib file was populated from ADS. To
%% get the citations to show in the compiled file do the following:
%%
%% pdflatex sample7.tex
%% bibtext sample7
%% pdflatex sample7.tex
%% pdflatex sample7.tex

\bibliography{Bibliography}
\bibliographystyle{Config/NoMod/aasjournalv7.bst}
% \bibliographystyle{authortitle}


% ================================================ APPENDIX =====================================================
\appendix

% \section{Appendix information}

Appendices can be broken into separate sections just like in the main text.
The only difference is that each appendix section is indexed by a letter
(A, B, C, etc.) instead of a number.  Likewise numbered equations have
the section letter appended.  Here is an equation as an example.
\begin{equation}
I = \frac{1}{1 + d_{1}^{P (1 + d_{2} )}}
\end{equation}
Appendix tables and figures should not be numbered like equations. Instead
they should continue the sequence from the main article body.
% \section{Author publication charges} \label{sec:pubcharge}

In April 2011 the traditional way of calculating author charges based on 
the number of printed pages was changed.  The reason for the change
was due to a recognition of the growing number of article items that could not 
be represented in print. Now author charges are determined by a number of
digital ``quanta''.  A single quantum is defined as 350 words, one figure, one table,
and one digital asset.  For the latter this includes machine readable
tables, data behind a figure, figure sets, animations, and interactive figures.  The current cost
for the different quanta types is available at 
\url{https://journals.aas.org/article-charges-and-copyright/#author_publication_charges}. 
Authors may use the ApJL length calculator to get a {\tt rough} estimate of 
the number of word and float quanta in their manuscript. The calculator 
is located at \url{https://authortools.aas.org/ApJL/betacountwords.html}.
\section{Rotating tables} \label{sec:rotate}

To place a single page table in a landscape mode start the table portion with {\tt\string\begin\{rotatetable\}} and end with {\tt\string\end\{rotatetable\}}.

Tables that exceed a print page take a slightly different environment since both rotation and long table printing are required. In these cases start with {\tt\string\begin\{longrotatetable\}} and end with {\tt\string\end\{longrotatetable\}}. The {\tt\string\movetabledown} command can be used to help center extremely wide, landscape tables. The command {\tt\string\movetabledown=1in} will move any rotated table down 1 inch. 

A handy "cheat sheet" that provides the necessary \latex\ to produce 17
different types of tables is available at \url{http://journals.aas.org/authors/aastex/aasguide.html#table_cheat_sheet}.

% \section{Using Chinese, Japanese, and Korean characters}

Authors have the option to include names in Chinese, Japanese, or Korean (CJK)
characters in addition to the English name. The names will be displayed
in parentheses after the English name. The way to do this in AASTeX is to
use the CJK package available at \url{https://ctan.org/pkg/cjk?lang=en}.
Further details on how to implement this and solutions for common problems,
please go to \url{https://journals.aas.org/nonroman/}.

%% This command is needed to show the entire author+affiliation list when
%% the collaboration and author truncation commands are used.  It has to
%% go at the end of the manuscript.
%\allauthors

%% Include this line if you are using the \added, \replaced, \deleted
%% commands to see a summary list of all changes at the end of the article.
%\listofchanges

\end{document}

% End of file `sample7.tex'.