\section{Introducción y Marco Teórico} \label{sec:intro}

\subsection{Energía} \label{subsec:energy}

\lipsum[1] \citep{resnick_physics}.

\subsubsection{Energía Cinética} \label{subsubsec:kinetic}

\lipsum[1] \ref{ec:kinetic_e}. Donde \(m\) es la masa del carro y \(v\) su velocidad \citep{resnick_physics}.

\begin{equation}
    \label{ec:kinetic_e}
    K = \frac{1}{2}mv^2
\end{equation}

\subsubsection{Energía Potencial Gravitatoria} \label{subsubsec:potential}

\lipsum[1][1] \ref{ec:mgh}. \lipsum[1][1] (\(m\)) y la altura (\(h\)) del cuerpo \citep{Kleppner}.  

\begin{equation}
    \label{ec:mgh}
    U = mgh
\end{equation}

\lipsum[1][2] \citep{resnick_physics}.

\subsubsection{Energía Mecánica} \label{subsubsec:mechanic_e}

\lipsum[1][3] \ref{ec:total_e} \citep{resnick_physics}.

\begin{equation}
    \label{ec:total_e}
    E = K + U
\end{equation}

\lipsum[1][1]

\subsubsection{Conservación de la Energía} \label{subsubsec:energy_conservation}

\lipsum[1][1] \citep{Kleppner}.

\begin{equation}
    \label{ec:conservation_e}
    K_i + U_i = K_f + U_f = E
\end{equation}

\lipsum[1][1]

\begin{figure}[ht!]
    \centering
    \includegraphics[width=8.5cm]{Resources/img/energy_linear_unitless.png}
    \caption{Ejemplo de conservación de la energía. Los datos se encuentran normalizados.}
    \label{fig:energy_linear}
\end{figure}

\subsection{Fuerzas} \label{subsec:forces}

\subsubsection{Teorema Trabajo-Energía} \label{subsubsec:work_energy_teo}

\lipsum[1] \citep{Kleppner}. Este teorema permite analizar sistemas dinámicos considerando únicamente las fuerzas que actúan sobre estos.

\begin{equation}
    \label{ec:work_energy}
    W_{b\to a} = \cint^{\hspace{0.1cm}a}_{\hspace{-0.2cm}b} \mathbf{F} \cdot \diff \mathbf{r}
\end{equation}

\begin{equation}
    \label{ec:work_energy_2}
    W_{b \to a} = \Delta E
\end{equation}

\lipsum[1][1]


\subsubsection{Movimiento Circular} \label{subsubsec:circular_mov}

\lipsum[2]
\begin{equation}
    \label{ec:c_accel}
    \mathbf{a}_c = \frac{v^2}{r} \, \hat{\mathbf{r}}
\end{equation}

\lipsum[1]

\begin{equation}
    \label{ec:f_c}
    \mathbf{F}_c = m \mathbf{a}_c = m \frac{v^2}{r} \, \hat{\mathbf{r}}
\end{equation}

\lipsum[1]

\begin{equation}
\label{ec:f_cf}
    \mathbf{F}_{cf} = - m \frac{v^2}{r} \, \hat{\mathbf{r}}
\end{equation}

 